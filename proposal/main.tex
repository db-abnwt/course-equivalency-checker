\documentclass{report}

\usepackage{nwstyle}

\author{Nawat Ngerncham and Chayapol Bunnag}

\title{Project Proposal}

\begin{document}
\maketitle

\section*{Introduction}
Currently, the MUIC International Affairs unit which deals with
exchange programs has a \href{https://sites.google.com/mahidol.edu/icabroad/home}
{website} that contains the information for any MUIC student
interested in doing an exchange program. It also has a link
to another \href{https://datastudio.google.com/u/0/reporting/99a044ff-1412-494e-a8b4-eb8a821bbc53/page/X3alC}
{website} that contains the information about course equivalency
between MUIC courses and partner universities' courses.
However, these websites are separated and both lack certain
good-to-have features. Thus, we would like to propose for a
rework of the IC Abroad website that contains information
about partner universities, course equivalency, and more such
that it would be more beneficial to students.

\section*{Problem Definition}
This problem is in the education domain since
it concerns MUIC students who wants to go on exchange. However,
the current \href{https://sites.google.com/mahidol.edu/icabroad/home}
{website} contains the following pain points:

\begin{itemize}
    \item The available exchange slots on the
        \href{https://sites.google.com/mahidol.edu/icabroad/application-for-exchange-program}
        {Application for exchange program page} show every single
        partner university, which is too much and has no obvious
        way to search by country or your target university unless
        `Find in page' was used.
    \item The list of partner universities is very difficult to 
        navigate and search for a specific university
    \item The partner university page doesn't show the approved
        equivalent courses anymore, and just gives a link to the 
        Google Data Studio \href{https://datastudio.google.com/u/0/reporting/99a044ff-1412-494e-a8b4-eb8a821bbc53/page/X3alC}
        {website} for course equivelency, which is annoying since
        they are separate websites.
    \item The full course equivalency \href{https://datastudio.google.com/u/0/reporting/99a044ff-1412-494e-a8b4-eb8a821bbc53/page/X3alC}
        {website} cannot be searched by major without doing some
        intense ticking.
\end{itemize}

\section*{Features and Functionalities}
Following are the features we plan to have in our proposed website:
\begin{itemize}
    \item It will have the same pages as the original IC Abroad 
        website but with different designs and slightly different 
        navigation layout.
    \item Available exchange slots and list of partner university
        will be optimized and searchable.
    \item Partner university page will contain the approved 
        equivalent courses
    \item The course equivalency records will be searchable by
        major and will be part of the same website.
\end{itemize}

\section*{Stakeholders}
\begin{itemize}
    \item The International Affairs unit would be the one to
        manage the website once deployed, so they would be the
        administrative user.
    \item The students at MUIC would be the ones using the website
        in a mostly read-only mode where they can get information
        for their exchange program.
\end{itemize}

\section*{Responsibilities}
Both members will come up with the database schema together.
Nawat will come up with the basic design and layout template
of each page but each page will be worked on separately by
each member $-$ implementing both the HTML template and the
code that backs it up.

\end{document}